\documentclass{article}
\usepackage{polyglossia}
\setmainlanguage{spanish}
\usepackage{caption}
\usepackage{geometry}
\usepackage{listings}
\usepackage{array,makecell}
\usepackage[inline]{enumitem}
\usepackage{circuitikz}
\usepackage{color}
\usepackage[hidelinks]{hyperref}
\usepackage{url}

\author{Juan Francisco Cabrera Sánchez \and Carlos Gallardo Polanco \and \newline \url{https://github.com/gpcarlos95/Electrical-Circuits-Language}}
\title{Manual de Usuario}
\date{\today}


\lstdefinestyle{bash}{
    language=bash,
    basicstyle=\footnotesize,
    breakatwhitespace=false,
    breaklines=false,
    captionpos=b,
    keepspaces=true,
    numbersep=5pt,
    showspaces=false,
    showstringspaces=false,
    showtabs=false,
    tabsize=2
}


\begin{document}
\maketitle

\section{Introducción}

En primer lugar, se describen las palabras reservadas del lenguaje, que son las siguientes:

\

\begin{enumerate*}
  \item[$\ $]\textit{switch}
  \item[$\ $]\textit{button}
  \item[$\ $]\textit{lamp}
  \item[$\ $]\textit{bell}
  \item[$\ $]\textit{fuse}
  \item[$\ $]\textit{relay}
  \item[$\ $]\textit{minute}
  \item[$\ $]\textit{plug}
  \item[$\ $]\textit{lock}
  \item[$\ $]\textit{regulator}
  \item[$\ $]\textit{movDetector}
  \item[$\ $]\textit{R}
  \item[$\ $]\textit{S}
  \item[$\ $]\textit{G}
\end{enumerate*}

\

Todas estas palabras reservadas, a excepción de R,S y G, pueden además, ir acompañadas de caracteres numéricos. Por tanto, en este lenguaje, se definen las instrucciones de la siguiente forma:

$$ PalabraReservada \ ( PalabraReservada , \ PalabraReservada , ...) $$

Cabe destacar que, a diferencia de muchos lenguajes, aquí no se utiliza el ';' como separador de instrucciones, sino que, cada instrucción está separada por espacios en blanco o por saltos de línea indistintamente.

Como posiblemente conozca, en los circuitos eléctricos, existen diferencias entre el hecho de conectar varios componentes en serie o conectarlos en paralelo, como es el caso de las diferencias en la tensión o en la corriente.

Por ello, se ha buscado, que resulte sencillo diferenciar cuándo varios componentes se conectan en serie o en paralelo. En este caso, es tan sencilo como que, dados un conjunto de componentes: $C_1,C_2,C_3, C_4,...,C_n$, si todos están conectados al mismo componente por la izquierda por una parte, y al mismo componente por la derecha por otra. Análogamente, se puede describir fácilmente cuándo un conjunto de componentes está conectado en serie. Dado un conjunto de componentes: $C_1,C_2,C_3, C_4,...,C_n$, diremos que están conectados en serie siempre que el componente que esté a la izquierda de $C_{n}$ sea $C_{n-1}$ y el componente que esté a la derecha de $C_{n-1}$ sea $C_{n}$. La mejor forma de comprobar esto es con un ejemplo:

\begin{itemize}
  \item En serie:
  \begin{lstlisting}
    lamp(R,button1)
    button1(lamp,fuse2)
    fuse2(button1, switch3)
    switch3(fuse2, bell0)
  \end{lstlisting}

  \

  \item En paralelo:
  \begin{lstlisting}
    switch(R,S)
    lamp1(R,S)
    button1(R,S)
  \end{lstlisting}

\end{itemize}

Por tanto, hay 3 posibilidades, que el circuito sea un conjunto de componentes conectados en serie, que el circuito sea un conjunto de componentes conectados en paralelo o la más común, que el circuito sea una combinación de las dos anteriores. Esto se puede observar en el siguiente cuadro:
\begin{table}[h!]
  \centering
  \begin{tabular}{| >{\centering\arraybackslash}m{3.5cm}
    |>{\centering\arraybackslash}m{9cm}|}
    \hline
    Código & Output \\ \hline
\begin{lstlisting}
lamp0(R,lamp1)
lamp1(lamp0,S)
      \end{lstlisting} & \begin{circuitikz} \draw
       (0,0)
         to[short] (0,1)
         to[short] (0,-1)
          (0,0) node[label={[font=\footnotesize]left:R}] {}
             to[short,o-] (1,0)
             to[lamp,label={lamp0}] (2,0)
          (2,0) -- (4,0)
             to[lamp, -o, label={lamp1}] (6,0)
          (6,0)node[label={[font=\footnotesize]right:S}] {}
          (6,0)
            to[short](6,1)
            to[short](6,-1)
          ;
     \end{circuitikz}
     \\ \hline
     \begin{lstlisting}
lamp0(R,S)
lamp1(R,S)
           \end{lstlisting} &
           \begin{circuitikz} \draw
                (0,0)
                  to[short] (0,1)
                  to[short] (0,-1)
                (0,0) node[label={[font=\footnotesize]left:R}] {}
                  to[short,o-o] (1,0)

                 (1,0) -- (1,0)
                   to[lamp,o-o,label={lamp0}] (3,0)
                 (1,0) -- (1,1.5)
                   to[lamp,label={lamp1}] (3,1.5)
                 (3,1.5)
                   to[short](3,0)
                 (3,0)
                   to[short,-o](4,0)
                (4,0)node[label={[font=\footnotesize]right:S}] {}
                (4,0)
                  to[short](4,1)
                  to[short](4,-1)
                ;
           \end{circuitikz}
           \\ \hline
           \begin{lstlisting}
button(R,lamp0)
lamp0(button,S)
button1(R,lamp1)
lamp1(button,S)
                 \end{lstlisting} &
                 \begin{circuitikz} \draw
                   (0,0)
                     to[short] (0,1)
                     to[short] (0,-1)
                      (0,0) node[label={[font=\footnotesize]left:R}] {}
                         to[short,o-o] (1,0)
                       (1,0) -- (1,0)
                         to[push button,label=button] (3,0)
                         to[lamp,label={lamp0}] (4,0)
                       (1,0) -- (1,1.5)
                         to[push button,label=button1] (3,1.5)
                         to[lamp,label={lamp1}] (4,1.5)
                         to[short](5,1.5)
                       (5,1.5)
                         to[short,-o](5,0)
                       (4,0)
                         to[short,-o](6,0)
                      (6,0)node[label={[font=\footnotesize]right:S}] {}   (6,0)
                        to[short](6,1)
                        to[short](6,-1)
                      ;
                 \end{circuitikz}
                 \\ \hline
  \end{tabular}
  \caption{Resumen de la sintaxis y la salida que produce.}
  \label{tab:Resumen}
\end{table}

Se deben tener las siguientes consideraciones a la hora de programar: no se permite la redefinición de componentes, es decir, si se intenta definir dos veces el mismo componente, habrá un mensaje de error indicando qué componente está repetido, considerándose el circuito inválido por este motivo. Y por último, recuerde que el circuito debe quedar cerrado, si no lo está, se considerará inválido y se notificará con el mensaje de error correspondiente.


\

\section{Compilación y ejecución}

\lstset{style=bash}

A partir de aquí se le explicará cómo compilar su circuito, y se le dan dos opciones:
\begin{itemize}
\item Usando el makefile:
  \begin{lstlisting}
  Para compilarlo todo:
    make

  Para compilar el léxico:
    make flex

  Para compilar la semántica:
    make bison

  Para compilar, enlazar y generar un ejecutable:
    make analyzer

  Para ejecutar el analizador:
    make exe
\end{lstlisting}
\item Sin usar el makefile:
\begin{lstlisting}[basicstyle=\footnotesize]{language=bash}
  Para compilar el léxico:
    flex lexicon.l

  Para compilar la semántica:
    bison -d syntactic.y

  Para compilar, enlazar y generar un ejecutable:
    g++ lex.yy.c syntactic.tab.c -o analyzer -lfl -lm

  Para ejecutar el analizador:
    cat NombreDelCircuito.circuit | ./analyzer

\end{lstlisting}
\end{itemize}
\end{document}
